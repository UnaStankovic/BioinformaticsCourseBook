% hiperlinkovi - url i href
\usepackage{url}

% dodatne matematicke oznake
\usepackage{amssymb}
\usepackage{amsmath}
\usepackage{amsthm}

% jezicki paketi
\usepackage[serbian]{babel} % podrska za srpski jezik
\usepackage[utf8]{inputenc} % podrska za utf8 kodiranje

% izgled strane i boje
\usepackage[hmarginratio=1:1, bottom = 1.5in]{geometry}
\PassOptionsToPackage{svgnames}{xcolor}

% paketi za crtanje grafike
\usepackage{tikz}
\usetikzlibrary{intersections, calc, backgrounds, shapes.misc, arrows, petri, topaths, decorations.markings, automata, positioning}

% bojenje hiperlinkova
\usepackage[colorlinks=true, linkcolor=red!80, citecolor=green, urlcolor=blue]{hyperref}

% za podesavanje elemenata na strani
\usepackage{adjustbox}

% napredne figure i tabele
\usepackage{caption}
\usepackage{float}

% podesavanje hedera i futera
\usepackage{fancyhdr}	
\pagestyle{fancy}
\fancyhead{} 
\fancyhead[RO,RE]{\thepage}
\fancyhead[LO,LE]{\slshape \leftmark}
\fancyfoot{} 
\fancyfoot[C]{ }

% za napredne kodove - koristi se okruzenje "lstlisting"
\usepackage{listings}
\lstset{
	basicstyle=\ttfamily,
	columns=fullflexible,
	keepspaces=true,
	showstringspaces=false,
	escapechar=\%,
	tabsize=4,
	language=Python,
	breaklines=true,
	postbreak=\mbox{\textcolor{red}{$\hookrightarrow$}\space},
	frame=L,
   	keywordstyle=\bfseries\color{green!40!black},
   	commentstyle=\itshape\color{purple!40!black},
   	identifierstyle=\color{blue},
   	stringstyle=\color{orange}
}

% dodatna okruzenja
\newtheorem{definicija}{Definicija}[chapter]
\newtheorem{teorema}[definicija]{Teorema}
\newtheorem{lema}[definicija]{Lema}
\newtheorem{primer}[definicija]{Primer}
\newtheorem{zadatak}[definicija]{Zadatak}
\newtheorem*{dokaz}{Dokaz}

% dodatne komande
\newcommand{\blankpage}{\newpage\hbox{}\thispagestyle{empty}\newpage}
\renewcommand\qedsymbol{\hspace*{\stretch{1}} $\blacksquare$}