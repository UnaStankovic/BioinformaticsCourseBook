\chapter{Kako poredimo biološke sekvence?}
\setbookcodestyle

\section{Biološki uvid u poređenje sekvenci}

Kako su biološke sekvence podložne promeni, umetanju i brisanju, čest je slučaj da i-ti simbol jedne sekvence odgovara simbolu na drugoj poziciji druge sekvence. U tom slučaju, cilj je postići najbolje poklapanje simbola.
Na primer, $ATGCATGC$ i $TGCATGCA$ nemaju delove koji se poklapaju, pa je njihova Hamingova udaljenost 8:

\begin{center}
$ATGCATGC$\\
$TGCATGCA$
\end{center}
    
Ali ako ih malo drugačije poravnamo, ove dve niske imaju 6 poklapajucih pozicija:

\begin{center}
$A\textcolor{red}{TGCATGC}-$\\
$-\textcolor{red}{TGCATGC}A$
\end{center}

Stringovi ATGCTTA i TGCATTAA imaju manje uocljive slicnosti:

\begin{center}
$A\textcolor{red}{TGC}-\textcolor{red}{TTA}-$\\
$-\textcolor{red}{TGC}A\textcolor{red}{TTA}A$
\end{center}

Ovi primeri navode nas da definisemo dobro poravnanje kao ono koje ima najveći mogući broj poklapanja. Povećanje broja poklapanja simbola možemo posmatrati kao igricu u kojoj u svakom potezu imamo dva izbora. Možemo da uklonimo oba simbola i osvojimo poen ako su oni isti ili mozemo ukloniti simbol iz jedne od niski, ne osvojimo poene, ali omogućimo da u daljem igranju osvojimo više poena. Cilj je da maksimizujemo broj poena.


%%%%%%%%%%%%%%%%%%%%%%%% ALEX
\section{Igra poravnanja i najduža zajednička podsekvenca}

Kod \textbf{Igre poravnavanja} cilj je ukloniti sve simbole iz
sekvenci tako da pritom sakupimo što više poena :
\begin{itemize}
    \item Uklanjanje prvog simbola iz svake sekvence
        \begin{itemize}
            \item 1 poen ako se simboli poklapaju
            \item 0 ako se simboli ne poklapaju
        \end{itemize}
    \item Uklanjanje prvog simbola iz jedne sekvnce
         \begin{itemize}
            \item 0 poena
        \end{itemize}
\end{itemize}

\begin{figure}[h]
\centering
\includegraphics[width=\textwidth]{poglavlja/5/slike/igraPoravnavanja.png}
\caption{Igra poravnavanja}
\end{figure} 


\textbf{Poravnanje} dve sekvence predstavlja matricu koja ima dva reda:

\begin{enumerate}
    \item red: simboli prve sekvence (redom) eventualno sa ubačenim “-” 
    \item red: simboli druge sekvence (redom) eventualno sa ubačenim “-” 
\end{enumerate}
\begin{figure}[h]
\centering
\includegraphics[width=\textwidth]{poglavlja/5/slike/poravnanje.png}
\caption{Poravnanje}
\label{slika:poravnavanje}
\end{figure} 

\subsection{Najduža zajednička podsekvenca}

Poklapanja (matches) u poravnanju dve sekvence ( u primeru \ref{slika:poravnavanje} to je ATGT) formiraju njihovu zajedničku podsekvencu.

%%%%%%%%%%%%%%%%%%%%%%%%%%%%%%%%%%%%%%%%%%%%

\section{Problem turiste na Menhetnu}

\noindent Pre svega postavimo problem:\\

\begin{problem}[Problem turiste na Menhetnu]
	Naći najdužu putanju u pravougaonoj mreži gradskih ulica. \\
	Ulaz: Usmeren težinski mrežni graf. \\
	Izlaz: Najduža putanja od početnog (source) do krajnjeg čvora (sink) u mrežnom grafu. 
\end{problem}

\begin{figure}[h!]
\centering
\includegraphics[width=0.7\textwidth]{poglavlja/5/slike/menhetn2.png}
\caption{Problem turiste na Menhetnu}
\label{slika:menhetn}
\end{figure} 

Na slici \ref{slika:menhetn} grafički je prikazan problem turiste na Menhetnu. Cilj je stići od plavog do crvenog kruga i pri tom sakupiti što više poena. Dozvoljeno kretanje je dole i desno. Možemo koristiti pohlepni algoritam i tako doći do cilja, ali da li smo tako sakupili najviše poena?

\noindent Dodatna izmena grafa bi bila da imamo i dijagnalne grane (\ref{slika:menhetn3}).

\begin{figure}[h!]
\centering
\includegraphics[width=0.7\textwidth]{poglavlja/5/slike/menhetn3.png}
\caption{Nepravilna mreža}
\label{slika:menhetn3}
\end{figure}

\noindent Time dolazimo do sledećeg problema:

\begin{problem}[Problem najduže putanje u usmerenom grafu]
	Naći najdužu putanju između dva   čvora u težinskom usmerenom grafu. \\
	Ulaz: Usmereni težinski graf sa označenim
čvorovima source i sink. \\
	Izlaz: Najduža putanja od čvora source do čvora
sink u usmerenom težinskom grafu. 
\end{problem}

Ako se prisetimo igre poravnanja, videćemo da postoji veza između ova dva problema (igre).

\begin{figure}[h!]
\centering
\includegraphics[width=0.7\textwidth]{poglavlja/5/slike/menhetn4.png}
\caption{Poravnanje $\rightarrow$ Putanja}
\label{slika:menhetn4}
\end{figure}

Pitamo se kako izgraditi graf za igru poravnanja i za problem najduže podsekvence. To ćemo uraditi na sledeći način:

\begin{itemize}
    \item Vrste označimo aminokiselinama iz prve niske
    \item Kolone označimo aminokiselinama iz druge niske
    \item U svaku presečnu tačku postavimo jedan čvor
    \item Gde god je moguće, postaviti vertikalne (insercija), horizontalne (delecija) i dijagonalne grane (match ili mismatch)
    \item Dijagonalne grane otežati koeficijentom 1, ostale koeficijentom 0
    \item Problem najduže zajedničke podsekvence se svodi na problem nalaženja najduže putanje između dva data čvora u usmerenom grafu
\end{itemize}

Kada nađemo poravnanje najvišeg skora našli smo i najdužu putanju u mrežnom grafu.

\begin{figure}[h!]
\centering
\includegraphics[width=0.7\textwidth]{poglavlja/5/slike/poravnanje1.png}
\caption{Poravnanje $\rightarrow$ Putanja}
\label{slika:poravnanje2}
\end{figure}

Dijagonalne crvene grane odgovaraju poklapanju simbola i imaju skor 1 (\ref{slika:poravnanje2})

%%%%%%%%%%%%%%%%%%%%%%%%%%%%%%%%%%%%%jasmina%%%%%%%%%%%%%%%%%%%%%%%%%%%%%%%%%%%%%%%%%%%%%%%%%%%%%%%%%%%%%%%%%%%%%%%%%%%%%%%%%%%%%%%%%%%%%%%%%%%%%%%%
\section{Problem kusura}

\noindent Upoznajmo se sa sledećim problemom:\\

\begin{problem}[Problem vraćanja kusura]
	Naći minimalan broj novčića neophodnih za vraćanje kusura. \\
	$Ulaz: Ceo broj money i niz pozitivnih celih brojeva (coin_1, coin_2, ..., coin_d). \\$
	$Izlaz: Minimalan broj novčića (coin_1, coin_2, ..., coin_d) u apoenima koji rasitnjava sumu money.$
\end{problem}

\subsection{Pohlepni algoritam}

Najzastupljeniji način vraćanja kusura širom sveta podrazumeva iterativno traženje sledećeg najvećeg novčića.\\
To bi značilo da bismo za kusur od 42 dinara dobili sledeće novčiće: 20 + 10 + 10 + 2. \\
Ovakav način vraćanja kusura opisuje takozvani pohlepni algoritam. \\

\begin{lstlisting}
GreedyChange(money)
begin
    change %$\leftarrow$% empty collection of coins
	while money > 0
		coin %$\leftarrow$% largest denomination that does not exceed money
		add coin to change
		money %$\leftarrow$% money - coin
	return change
end
\end{lstlisting}

Međutim, ako malo bolje razmislimo ovo rešenje zapravo nije najbolje.\\
Kusur bismo mogli vratiti i sa manje novčića na sledeći način: 42 = 20 + 20 + 2 \\

\textit{Zaključak}: \textbf{GreedyChange ne daje optimalno rešenje!}

\subsection{Rekurzivni algoritam}

Pokušajmo sada da problem rešimo na drugačiji način koristeći rekurziju. \\
Za zadate apoene 6, 5, 1, koji je najmanji broj novčića neophodnih za vraćanje kusura od 9 centi? \\

\begin{figure}[h!]
\centering
\includegraphics[width=0.7\textwidth]{poglavlja/5/slike/rekurzija1.JPG}
\caption{Vracanje kusura - rekurzija}
\label{slika:rekurzija}
\end{figure}

Problem resavamo tako sto prvo od 9 oduzmemo 6 i dobijemo 3 kao ostatak kusura. Dakle, 9 se može vratiti od jednog novčića od 6 apoena i jos plus broj novčiča koji je potreban za preostali deo kusura od 3 centa. \\
U istoj iteraciji analogno računamo za preostale apoene.\\

Na slici \ref{slika:rekurzija} crvenim znakom pitanja označeno je traženo rešenje koje dobijemo rešavanjem manjih problema za kusure 3, 4 i 8. \\ 

MinNumCoins(9) = $\min$ 
$\begin{cases}$
$MinNumCoins(9-6) + 1 = MinNumCoins(3) + 1\\$
$MinNumCoins(9-5) + 1 = MinNumCoins(4) + 1\\$
$MinNumCoins(9-1) + 1 = MinNumCoins(8) + 1\\$
$\end{cases}$

Na osnovu prethodnog, moguće je izvesti opštu formulu:

MinNumCoins(money) = $\min$ 
$\begin{cases}$
$MinNumCoins(money-coin_{1}) + 1 \\$
$\dots \\$
$MinNumCoins(money-coin_{d}) + 1 \\$
$\end{cases}$
\\
Hajde sada da vidimo kako bismo to isprogramirali:
\\
\begin{lstlisting}
RecursiveChange(money, coins)
begin
    if money = 0
        return 0
    MinNumCoins %$\leftarrow$% infinity 
    for i %$\leftarrow$% 1 to |coins|
        if money %$\geq coin_{i}$%
            NumCoins %$\leftarrow$% RecursiveChange(money - %$coin_{i}$%, coins)
            if numCoins + 1 < MinNumCoins
                MinNumCoins %$\leftarrow$% numCoins + 1
	return MinNumCoins
end
\end{lstlisting}

Reklo bi se da smo sada dobili odgovarajuci algoritam za naš problem, hajde to da proverimo. \\
Postavlja se pitanje, koliko je brz RecursiveChange? \\

Pokušajmo na konkretnom primeru da dođemo do rešenja. Neka naš problem sada bude vraćanje kusura od 76 centi. Pomoću rekurzivnog stabla demonstrirajmo ponašanje našeg algoritma: \\

\begin{figure}[h!]
\centering
\includegraphics[width=0.7\textwidth]{poglavlja/5/slike/rekurzivnoStablo.JPG}
\caption{Vracanje kusura - ponašanje rekurzivnog algoritma}
\label{slika:rekurzija2}
\end{figure}

Ono što se odmah može primetiti jeste višestruko pozivanje algoritma za vrednost od 69 centi, čak 6 puta! \\
Daljim procenama možemo doći do zaključka da se optimalna kombinacija novčića za 30 centi izračunava milijardama puta! \\

Sada je očigledno da nam rekurzija ne rešava problem na najbolji mogući način. \\

\subsection{Vraćanje kusura dinamičkim programiranjem}

Cilj nam je da izbegnemo višestruka izračunavanja  vraćanja kusura za istu vrednost, tako da bi ideja bila da imamo objekat koji će pamtiti sva računanja i iz koga ćemo čitati već izračunate vrednosti. \\
Dakle, umesto vremenski zahtevnih poziva \\
\begin{center}
RecursiveChange(money - $coin_{i}$, coins)
\end{center}
jednostavno bismo potražili vrednosti iz unapred izračunate tabele \\
\begin{center}
MinNumCoins(money - $coin_{i}$).
\end{center}


\begin{lstlisting}
DPChange(money, coins)
begin
    MinNumCoins(0) %$\leftarrow$% 0 
    for m %$\leftarrow$% 1 to money
        MinNumCoins(m) %$\leftarrow$% infinity
        for i %$\leftarrow$% 1 to |coins|
            if m %$\geq coin_{i}$%
                if MinNumCoins(m - %$coin_{i}$%) + 1 < MinNumCoins(m)
                    MinNumCoins(m) %$\leftarrow$% MinNumCoins(m - %$coin_{i}$%) + 1
	return MinNumCoins(money)
end
\end{lstlisting}


\section{Dinamičko programiranje i putokazi za povratak}

Posmatramo jednostavniji, Menhetn graf:
Pretpostavimo da do čvora sink možemo doći samo na dva načina: kretanjem južno $\downarrow$ ili kretanjem istočno $\rightarrow$

\begin{figure}[h!]
\centering
\includegraphics[width=0.7\textwidth]{poglavlja/5/slike/putokazi.png}
\caption{Južno ili istočno?}
\label{slika:putokazi}
\end{figure}

\noindent Prvo probamo da rešimo problem rekurzivno:

\begin{lstlisting}
SouthOrEast(n, m)
if n=0 and m=0
    return 0
x %$\leftarrow$% -infinity, y %$\leftarrow$% -infinity
if n > 0
x %$\leftarrow$% SouthOrEast(n-1,m)+weight of edge "%$\downarrow$%" into (n, m)
if m > 0
y %$\leftarrow$% SouthOrEast(n,m-1)+ weight of edge "%$\rightarrow$%" into (n,m)
return max{x, y}
\end{lstlisting}

\noindent Ovaj algoritam se poziva za svaki čvor u grafu veličine $m\times n$, a pri tom se dešava da za jedan isti čvor računamo više puta. Zbog toga je ovaj pristup previše spor, pa prelazimo na dinamičko programiranje.
Krenućemo  od početnog čvora. Zatim, u čvor (i, j) upisujemo dužinu maksimalne putanje od (0,0) do (i,j). Prvo izračunamo za čvorove na obodu grafa a zatim, kolonu po kolonu, za preostale čvorove.

\begin{figure}[h!]
\centering
\includegraphics[width=0.5\textwidth]{poglavlja/5/slike/putokazi1.png}
\caption{Južno ili istočno?}
\label{slika:putokazi1}
\end{figure}

\begin{figure}[h!]
\centering
\includegraphics[width=0.5\textwidth]{poglavlja/5/slike/putokazi2.png}
\caption{Južno ili istočno?}
\label{slika:putokazi2}
\end{figure}

\begin{figure}[h!]
\centering
\includegraphics[width=0.5\textwidth]{poglavlja/5/slike/putokazi3.png}
\caption{Južno ili istočno?}
\label{slika:putokazi3}
\end{figure}

Na slici \ref{slika:putokazi3} prikazane su podebljane grane koje predstavljaju putokaze za povratak od čvora sink do čvora source.

\subsection{Rekurentna relacija dinamičkog programiranja kod Menhetn grafa}

$s_i,_j$: the length of a longest path from (0,0) to (i,j)

$s_i,_j$ = $\max$ $\begin{cases}$
$s_{i-1},_j + weight of edge "\downarrow"into (i,j)\\$
$s_i,_{j-1} + weight of edge "\rightarrow"into (i,j)$
$\end{cases}$

\begin{lstlisting}
ManhattanTourist(n, m, Down, Right)
%$s_0,_0$% %$\leftarrow$% 0
for i %$\leftarrow$% 1 to n
    %$s_i,_0$% %$\leftarrow$% %$s_{i-1},_0$% + %$down_i,_0$%
for j %$\leftarrow$% 1 to m
    %$s_0,_j$% %$\leftarrow$% %$s_0,_{j-1}$% + %$right_0,_j$% 
for i %$\leftarrow$% 1 to n
    for j %$\leftarrow$% 1 to m
        %$s_i,_j$% %$\leftarrow$% %$\max$% { %$s_{i-1},_j$% + %$down_i,_j$%, %$s_i,_{j-1}$% + %$right_i,_j$% }
return %$s_n,_m$%
\end{lstlisting}



%%%%%%%%%%%%%%%%%%%%%%%%%%%%%%%%%%%%%%%%%%%%%%%%%%%%%%%%%%%%%%%%%%%%%%%%%

\section{Od Menhetna do grafa poravnanja }

\subsection{Rekurentna relacija dinamičkog programiranja kod grafa poravnanja}

Gledajući sliku \ref{slika:povratak} možemo da vidimo da se najduži put od (0,0) do (i,j) računa:


$s_i,_j$ = $\max$ $\begin{cases}$
$s_{i-1},_j$ + 
$\textcolor{green}{weight of edge "\downarrow"into (i,j)}\\$
$s_i,_{j-1}$ + 
$\textcolor{blue}{weight of edge "\rightarrow"into (i,j)}\\$
$s_{i-1},_{j-1}$ + 
$\textcolor{red}{weight of edge "\searrow"into (i,j)}$
$\end{cases}$

Što dalje daje :

$s_i,_j$ = $\max$ $\begin{cases}$
$\textcolor{green}{s_{i-1},_j + 0}\\$
$\textcolor{blue}{s_i,_{j-1} + 0}\\$
$\textcolor{red}{s_{i-1},_{j-1} + 1, v_i = w_j}\\$
$\textcolor{red}{s_{i-1},_{j-1} + 0, v_i \neq w_j}$
$\end{cases}$



\begin{figure}[]
\centering
\includegraphics[width=0.5\textwidth]{poglavlja/5/slike/graf1.png}
\caption{ Crvene grane težina 1, ostale grane težina 0, $v_i$ i $w_j$ oznake vrste i kolone}
\label{slika:povratak}
\end{figure}


U slici \ref{slika:backtrack} se vide boldovane grane koje su nastale primenom pravila rekurentne relacije. One predstavljaju putokaze za povratak (backtrack) kod grafa za najdužu zajedničku podsekvencu. 

\begin{figure}[]
\centering
\includegraphics[width=0.5\textwidth]{poglavlja/5/slike/backtrack.png}
\caption{Putokazi za povratak (backtrack)}
\label{slika:backtrack}
\end{figure}

\subsection{Računanje putokaza za povratak }

%% FORMULA!!

Podsetimo se sada kako bismo rekontruisali putanju preko putokaza kod Menhetn grafa? Krenuli bismo od krajnjeg čvora (sink) i pratili putokaze u obrnutom smeru do početnog čvora (source) \ref{slika:rek3}.
\begin{figure}[]
\centering
\includegraphics[width=0.7\textwidth]{poglavlja/5/slike/rek3.png}
\caption{Rekonstrukcija putanje preko putokaza kod Menhetn grafa}
\label{slika:rek3}
\end{figure}



\subsection{Određivanje najduže zajedničke podsekvence (LCS – longest common subsequence) korišćenjem putokaza za povratakc}

\begin{lstlisting}
OutputLCS (backtrack, v, i, j)
if i = 0 or j = 0
    return
if %$backtrack_i,_j$% = "%$\rightarrow$%"
    OutputLCS (backtrack, v, i, j-1)
else if %$backtrack_i,_j$% = "%$\downarrow$%"
    OutputLCS (backtrack, v, i-1, j)
else
    OutputLCS (backtrack, v, i-1, j-1)
    output  %$v_i$%
\end{lstlisting}

Do sada smo pretpostavljali da graf u kom tražimo
najdužu putanju ima samo tri vrste grana. Da li se OutputLCS može generalizovati tako da važi i za grafove koji nemaju tako specifičnu
topologiju?\\
Kako se rekurentna relacija dinamičkog programiranja menja za ovakav graf? \ref{slika:rekRel}
\begin{figure}[]
\centering
\includegraphics[width=\textwidth]{poglavlja/5/slike/rekurentaRelDinProg.png}
\caption{}
\label{slika:rekRel}
\end{figure}
\\

\textit{$s_a$ = $max_{all predecessors b of node a}{s_b+ weight of edge from b to a}$}
\\

Računanje skora za SVE prethodnike \ref{slika:racunanje}.

\begin{figure}[]
\centering
\includegraphics[width=0.5\textwidth]{poglavlja/5/slike/racunanje.png}
\caption{Začarani krug}
\label{slika:racunanje}
\end{figure}


\begin{itemize}
    \item Kod ovakve rekurentne relacije, važno je da pri računanju $s_a$ imamo izračunate $s_b$ za sve čvorove prethodnike b (čvorovi za koje
postoji grana do čvora a) Da li je to moguće u bilo kom usmerenom
težinskom grafu? Odgovor je \textbf{nije}. Da bismo kod svakog čvora mogli da mogli da izračunamo skor za sve njegove prethodnike, usmereni težinski graf mora biti acikličan. DAG (Directed Acyclic Graph) 
    \item Ako je dat usmereni aciklični graf, da li njegove čvorove možemo poređati u niz tako da njihov redosled u nizu osigurava uslov da pri računanju $s_a$ imamo izračunate $s_b$ za sve čvorove prethodnike b (čvorovi za koje postoji grana do čvora a)? Odgovor je \textbf{da}, moguće je poređati sve čvorove grafa u niz i taj niz topološki sortirati
\end{itemize}




\subsection{Topološko sortiranje}

\begin{itemize}
    \item \textbf{Topološko sortiranje} : Sortiranje čvorova DAG-a u nizu tako da sve grane u takvom nizu idu s leva na desno.
    \item \textbf{ Teorema}: Svaki DAG se može topološki sortirati.
    \item Topološko sortiranje svakog DAG-a se obavlja za O(edges) koraka.
\end{itemize}

\textbf{Algoritam za nalaženje najduže putanje u DAG-u>} :

\begin{lstlisting}
LongestPath(Graph, source, sink)
for each node a in Graph
     %$s_a$% %$\leftarrow$% -infinity
%$s_source$% %$\leftarrow$%  0
topologically order Graph
for each node a (from source to sink in topological order)
%$s_a$%  %$\leftarrow$% %$max_{all predecessors b of node a}$% {%$s_b$% + weight of edge from b to a}
return %$s_{sink}$%
\end{lstlisting}


\begin{itemize}
    \item Pošto svaka grana učestvuje tačno jednom, složenost je O(edges)
    \item LongestPath vraća dužinu najdužeg zajedničkog podniza ali ne rekonstruiše putanju
\end{itemize}


\section{Od globalnog do lokalnog poravnanja}



%%%%%%%%%%%%%%%%%%%%%%%%%%%%%%%%%%%%%%%%%%%%%%%%%%%%%%%%%%%%%%%%%%%%
\section{Kažnjavanje insercija i delecija u poravnanju sekvenci}

\subsection{Kažnjavanje praznina}

\begin{itemize}
    \item U globalnom poravnanju je fiksna kazna $\sigma$ bila dodeljena svakom indelu
    \item Međutim, ova fiksna kazna može biti preoštra
kod lokalnog poravnanja kada možemo imati 100 uzastopnih indela.
    \item Niz od k uzastopnih indela često predstavlja jedan isti evolucioni događaj, ne k različitih, slika \ref{slika:kaznjavanje}
\end{itemize}

\begin{figure}[]
\centering
\includegraphics[width=0.5\textwidth]{poglavlja/5/slike/kaznjavanjePraznina.png}
\caption{}
\label{slika:kaznjavanje}
\end{figure}

\subsection{Adekvatnije kazne za praznine}

Afina kazna za praznine za prazninu dužine k:  \textbf{$\sigma+\epsilon*(k-1)$}
\begin{itemize}
    \item $\sigma$ kazna za \textbf{otvaranje} praznine
    \item $\epsilon$ kazna za \textbf{proširenje} praznine
    \item $ \sigma > \epsilon$ , jer otvaranje praznine treba kazniti više nego njeno proširenje
\end{itemize}



U slici \ref{slika:modelovanje} prikayano je modelovanje afinih kazni za praznine pomoću dugih grana.

\begin{figure}[]
\centering
\includegraphics[width=0.7\textwidth]{poglavlja/5/slike/modelovanjePomocuDugihGrana.png}
\caption{Modelovanje afinih kazni za praznine pomoću dugih grana}
\label{slika:modelovanje}
\end{figure}

\subsection{Izgradnja Menhetn grafa sa afinim kaznama za praznine}

\begin{figure}[]
\centering
\includegraphics[width=0.5\textwidth]{poglavlja/5/slike/MenhettnAifnePraznine.png}
\caption{Dodali smo $O(n^3)$ grana}
\label{slika:afiniMenhetn}
\end{figure}


\begin{itemize}
    \item Vremenska složenost je direktno proporcionalna broju grana, zbog čega želimo da smanjimo broj grana u grafu (trenutno je
$O(n^3)$ \ref{slika:afiniMenhetn})
\item Jedan način za smanjivanje broja grana je povećanje broja čvorova u grafu
\item Zato delimo Menhetn graf na tri nivoa \ref{slika:triNivoa}
\end{itemize}


\begin{figure}[]
\centering
\includegraphics[width=\textwidth]{poglavlja/5/slike/triNivoa.png}
\caption{Podela Menhetna grafa na 3 nivoa}
\label{slika:triNivoa}
\end{figure}

Ako imamo putanju poput one na slici \ref{slika:kakoSimulirati}, kako da je predsavimo pomoću Menhetn grafa na 3 nivoa? Rešenje je prikazano na slici \ref{slika:simulacija}


\begin{figure}[]
\centering
\includegraphics[width=0.5\textwidth]{poglavlja/5/slike/kakoSimulirari.png}
\caption{Kako predstaviti pomoću Menhetn grafa na 3 nivoa?}
\label{slika:kakoSimulirati}
\end{figure}


\begin{figure}[]
\centering
\includegraphics[width=0.5\textwidth]{poglavlja/5/slike/simulacija.png}
\caption{Simulacija Menhetn grafa na 3 nivoa}
\label{slika:simulacija}
\end{figure}

%%%%%%%%%%%%%%%%%%%%%%%%%%%%%%%%%jasminaaaa%%%%%%%%%%%%%%%%%%%%%%%%%%%%%%%%%%%

\section{Prostorno efikasno poravnanje sekvenci}

Zapitajmo se sledeće: \\
Da li možemo poravnati NPR sintetaze iz dve rayličite bakterije? \\
\\

Uzmimo u obzir sledeće činjenice:
\begin{itemize}
    \item NPR sintetaze su obično veoma dugi proteini, približno 20 000 aminokiselina
    \item Vremenska složenost poravnanja je ~ #edges, odnosno kvadratna
    \item Prostorna složenost poravnanja je ~ #nodes, odnosno kvadratna
    \item \textbf{Memorija je često usko grlo pri poređenju dugih sekvenci}
\end{itemize}

%%%%%%%%%%%%%%%%%%%%%%%%%%%%%%%%%%%%%%%%%%%%%%%%%%%%%%%%%%%%%%%%%%%%

\section{Višestruko poravnanje sekvenci}

\subsection{Od dvostrukog do višestrukog poravnanja}
\begin{itemize}
    \item Do sada su u poravnanju učestvovale samo dve sekvence.
    \item Slaba sličnost između dve sekvence postaje značajna ako je prisutna i u drugim sekvencama
    \item Višestruka poravnanja mogu otkriti suptilne sličnosti koje dvostruka poravnanja ignorišu
\end{itemize}

\subsection{Poravnanje tri A-domena}
Na slici \ref{slika:poravnavanjaTriA} je prikazano poravnanje tri A-domena
\begin{figure}[]
\centering
\includegraphics[width=0.8\textwidth]{poglavlja/5/slike/poravnavanjaTriAdomena.png}
\caption{Poravnavanja 3 A-domena}
\label{slika:poravnavanjaTriA}
\end{figure}

\subsection{Generalizacija dvostrukog na višestroko poravnanje}

\begin{itemize}
    \item Poravnanje 2 sekvence je matrica od 2 reda
    \item Poravnanje 3 sekvence je matrica od 3 reda \ref{slika:poravnavanjeMatrica}
    \begin{figure}[]
    \centering
    \includegraphics[width=0.5\textwidth]{poglavlja/5/slike/poravnanjeMatrica.png}
    \label{slika:poravnavanjeMatrica}
    \end{figure}
    \item Funkcija skora treba da dodeljuje visok skor poravnanjima sa konzerviranim kolonama
\end{itemize}

\subsection{Poravnanja = 3-D putanje}

Poravnanje sekvenci ATGC, AATC i ATGC \ref{slika:3d}

\begin{figure}[]
\centering
\includegraphics[width=0.8\textwidth]{poglavlja/5/slike/3dPoravnanja.png}
\caption{Poravnanje sekvenci ATGC, AATC i ATGC}
\label{slika:3d}
\end{figure}

\subsection{2-D poravnanje u odnosu na 3-D poravnanje}
%%%%%%%%%%%%%%%%%%%%%%%%%%%%%%%%%%%%%%%%%%%%%%%%%%%%%%%%%%%%%%%%%%%

\newpage
\section{Zadaci sa vežbi}
\setexamplecodestyle

U nastavku će biti predstavljeni zadaci sa vežbi na kursu rađeni u programskom jeziku Python.

\subsection{Manhattan Tourist}

\lstinputlisting[language=Python]{poglavlja/5/kodovi/ManhattanTourist.py}

\subsection{LCS Backtrack}

\lstinputlisting[language=Python]{poglavlja/5/kodovi/LCSBacktrack.py}

\subsection{Global Alignment}

\lstinputlisting[language=Python]{poglavlja/5/kodovi/GlobalAlignment.py}

\subsection{Local Alignment}

\lstinputlisting[language=Python]{poglavlja/5/kodovi/LocalAlignment.py}

\subsection{Edit Distance}

\lstinputlisting[language=Python]{poglavlja/5/kodovi/EditDistance.py}